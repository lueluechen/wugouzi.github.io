% TIPS
% \substack{a\\b} for multiple lines text





% pdfplots will load xolor automatically without option
\usepackage[dvipsnames]{xcolor}

\usepackage{forest}
% two-line text in node by [two \\ lines]
% \begin{forest} qtree, [..] \end{forest}
\forestset{
  qtree/.style={
    baseline,
    for tree={
      parent anchor=south,
      child anchor=north,
      align=center,
      inner sep=1pt,
    }}}
%\usepackage{flexisym}
% load order of mathtools and mathabx, otherwise conflict overbrace

\usepackage{mathtools}
%\usepackage{fourier}
\usepackage{pgfplots}
\usepackage{amsthm}
\usepackage{amsmath}
%\usepackage{unicode-math}
%
\usepackage{commath}
%\usepackage{,  , }
\usepackage{amsfonts}
\usepackage{amssymb}
% importing symbols https://tex.stackexchange.com/questions/14386/importing-a-single-symbol-from-a-different-font
%mathabx change every symbol
% use instead stmaryrd
%\usepackage{mathabx}
\usepackage{stmaryrd}
\usepackage{empheq}
\usepackage{tikz}
\usepackage{tikz-cd}
%\usepackage[notextcomp]{stix}
\usetikzlibrary{arrows.meta}
\usepackage[most]{tcolorbox}
%\utilde
%\usepackage{../../latexpackage/undertilde/undertilde}
% left and right superscript and subscript
\usepackage{actuarialsymbol}
\usepackage{threeparttable}
\usepackage{scalerel,stackengine}
\usepackage{stackrel}
% \stackrel[a]{b}{c}
\usepackage{dsfont}
% text font
\usepackage{newpxtext}
%\usepackage{newpxmath}

%\newcounter{dummy} \numberwithin{dummy}{section}
\newtheorem{dummy}{dummy}[section]
\theoremstyle{definition}
\newtheorem{definition}[dummy]{Definition}
\newtheorem{corollary}[dummy]{Corollary}
\newtheorem{lemma}[dummy]{Lemma}
\newtheorem{proposition}[dummy]{Proposition}
\newtheorem{theorem}[dummy]{Theorem}
\theoremstyle{definition}
\newtheorem{example}[dummy]{Example}
\theoremstyle{remark}
\newtheorem*{remark}{Remark}


\newcommand\what[1]{\ThisStyle{%
    \setbox0=\hbox{$\SavedStyle#1$}%
    \stackengine{-1.0\ht0+.5pt}{$\SavedStyle#1$}{%
      \stretchto{\scaleto{\SavedStyle\mkern.15mu\char'136}{2.6\wd0}}{1.4\ht0}%
    }{O}{c}{F}{T}{S}%
  }
}

\newcommand\wtilde[1]{\ThisStyle{%
    \setbox0=\hbox{$\SavedStyle#1$}%
    \stackengine{-.1\LMpt}{$\SavedStyle#1$}{%
      \stretchto{\scaleto{\SavedStyle\mkern.2mu\AC}{.5150\wd0}}{.6\ht0}%
    }{O}{c}{F}{T}{S}%
  }
}

\newcommand\wbar[1]{\ThisStyle{%
    \setbox0=\hbox{$\SavedStyle#1$}%
    \stackengine{.5pt+\LMpt}{$\SavedStyle#1$}{%
      \rule{\wd0}{\dimexpr.3\LMpt+.3pt}%
    }{O}{c}{F}{T}{S}%
  }
}

\newcommand{\bl}[1] {\boldsymbol{#1}}
\newcommand{\Wt}[1] {\stackrel{\sim}{\smash{#1}\rule{0pt}{1.1ex}}}
\newcommand{\wt}[1] {\widetilde{#1}}
\newcommand{\tf}[1] {\textbf{#1}}


%For boxed texts in align, use Aboxed{}
%otherwise use boxed{}

\DeclareMathSymbol{\widehatsym}{\mathord}{largesymbols}{"62}
\newcommand\lowerwidehatsym{%
  \text{\smash{\raisebox{-1.3ex}{%
    $\widehatsym$}}}}
\newcommand\fixwidehat[1]{%
  \mathchoice
    {\accentset{\displaystyle\lowerwidehatsym}{#1}}
    {\accentset{\textstyle\lowerwidehatsym}{#1}}
    {\accentset{\scriptstyle\lowerwidehatsym}{#1}}
    {\accentset{\scriptscriptstyle\lowerwidehatsym}{#1}}
}

\usepackage{graphicx}
    
% text on arrow for xRightarrow
\makeatletter
%\newcommand{\xRightarrow}[2][]{\ext@arrow 0359\Rightarrowfill@{#1}{#2}}
\makeatother


\newcommand{\dom}[1]{%
\mathrm{dom}{(#1)}
}

\newcommand{\ran}[1]{%
  \mathrm{ran}{(#1)}
}



\newcommand{\trcl}[1]{%
  \mathrm{trcl}{(#1)}
}



% Roman numerals
\makeatletter
\newcommand*{\rom}[1]{\expandafter\@slowromancap\romannumeral #1@}
\makeatother

\def \fR {\mathfrak{R}}
\def \bx {\boldsymbol{x}}
\def \bz {\boldsymbol{z}}
\def \ba {\boldsymbol{a}}
\def \bh {\boldsymbol{h}}
\def \bo {\boldsymbol{o}}
\def \bU {\boldsymbol{U}}
\def \bc {\boldsymbol{c}}
\def \bL {\boldsymbol{L}}
\def \bV {\boldsymbol{V}}
\def \bI {\boldsymbol{I}}
\def \bK {\boldsymbol{K}}
\def \bt {\boldsymbol{t}}
\def \bb {\boldsymbol{b}}
\def \bA {\boldsymbol{A}}
\def \bX {\boldsymbol{X}}
\def \bu {\boldsymbol{u}}
\def \bS {\boldsymbol{S}}
\def \bZ {\boldsymbol{Z}}
\def \bz {\boldsymbol{z}}
\def \by {\boldsymbol{y}}
\def \bw {\boldsymbol{w}}
\def \bT {\boldsymbol{T}}
\def \bF {\boldsymbol{F}}
\def \bS {\boldsymbol{S}}
\def \bm {\boldsymbol{m}}
\def \bW {\boldsymbol{W}}
\def \bR {\boldsymbol{R}}
\def \bQ {\boldsymbol{Q}}
\def \bS {\boldsymbol{S}}
\def \bP {\boldsymbol{P}}
\def \bT {\boldsymbol{T}}
\def \bY {\boldsymbol{Y}}
\def \bH {\boldsymbol{H}}
\def \bB {\boldsymbol{B}}
\def \bG {\boldsymbol{G}}
\def \blambda {\boldsymbol{\lambda}}
\def \bPhi {\boldsymbol{\Phi}}
\def \btheta {\boldsymbol{\theta}}
\def \bTheta {\boldsymbol{\Theta}}
\def \bmu {\boldsymbol{\mu}}
\def \bphi {\boldsymbol{\phi}}
\def \bSigma {\boldsymbol{\Sigma}}
\def \lb {\left\{}
\def \rb {\right\}}
\def \la {\langle}
\def \ra {\rangle}
\def \caln {\mathcal{N}}
\def \dissum {\displaystyle\Sigma}
\def \dispro {\displaystyle\prod}
\def \E {\mathbb{E}}
\def \Q {\mathbb{Q}}
\def \N {\mathbb{N}}
\def \V {\mathbb{V}}
\def \R {\mathbb{R}}
\def \P {\mathbb{P}}
\def \A {\mathbb{A}}
\def \Z {\mathbb{Z}}
\def \I {\mathbb{I}}
\def \C {\mathbb{C}}
\def \cala {\mathcal{A}}
\def \calb {\mathcal{B}}
\def \calq {\mathcal{Q}}
\def \calp {\mathcal{P}}
\def \cals {\mathcal{S}}
\def \calg {\mathcal{G}}
\def \caln {\mathcal{N}}
\def \calr {\mathcal{R}}
\def \calm {\mathcal{M}}
\def \calc {\mathcal{C}}
\def \calf {\mathcal{F}}
\def \calk {\mathcal{K}}
\def \call {\mathcal{L}}
\def \calu {\mathcal{U}}
\def \calo {\mathcal{O}}
\def \calh {\mathcal{H}}
\def \bcup {\bigcup}

% set theory
\def \zfc {\textbf{ZFC }}
\def \zfcm {\textbf{ZFC}}
\def \zfcc {\textbf{ZFC}^-}
\def \ac  {\textbf{AC}}
\def \gl  {\textbf{L }}
\def \gll {\textbf{L}}
\newcommand{\zfm}{$\textbf{ZF}^-$}

%\def \zfm {$\textbf{ZF}^-$}
\def \zfmm {\textbf{ZF}^-}
\def \zf {\textbf{ZF }}
\def \wf {\textbf{WF }}
\def \on {\textbf{On }}
\def \cm {\textbf{M }}
\def \cn {\textbf{N }}
\def \cv {\textbf{V }}
\def \zc {\textbf{ZC }}
\def \zcm {\textbf{ZC}}
\def \zff {\textbf{ZF}}
\def \wfm {\textbf{WF}}
\def \onm {\textbf{On}}
\def \cmm {\textbf{M}}
\def \cnm {\textbf{N}}
\def \cvm {\textbf{V}}
\def \gchh {\textbf{GCH}}
\renewcommand{\restriction}{\mathord{\upharpoonright}}
\def \pred {\text{pred}}
\def \cl {\text{cl}}
\def \rank {\text{rank}}
\def \con {\text{Con}}
\def \deff {\text{Def}}


\def \uin {\underline{\in}}
\def \oin {\overline{\in}}
\def \uR {\underline{R}}
\def \oR {\overline{R}}
\def \uP {\underline{P}}
\def \oP {\overline{P}}

\def \Ra {\Rightarrow}

\def \e {\enspace}

\def \sgn {\operatorname{sgn}}
\def \gen {\operatorname{gen}}
\def \ker {\operatorname{ker}}
\def \im {\operatorname{im}}
\def \aut {\operatorname{Aut}}
\def \inn {\operatorname{Inn}}

\def \tril {\triangleleft}


% \varprod
\DeclareSymbolFont{largesymbolsA}{U}{txexa}{m}{n}
\DeclareMathSymbol{\varprod}{\mathop}{largesymbolsA}{16}

% \bigtimes
\DeclareFontFamily{U}{mathx}{\hyphenchar\font45}
\DeclareFontShape{U}{mathx}{m}{n}{
      <5> <6> <7> <8> <9> <10>
      <10.95> <12> <14.4> <17.28> <20.74> <24.88>
      mathx10
      }{}
\DeclareSymbolFont{mathx}{U}{mathx}{m}{n}
\DeclareMathSymbol{\bigtimes}{1}{mathx}{"91}
% \odiv
\DeclareFontFamily{U}{matha}{\hyphenchar\font45}
\DeclareFontShape{U}{matha}{m}{n}{
      <5> <6> <7> <8> <9> <10> gen * matha
      <10.95> matha10 <12> <14.4> <17.28> <20.74> <24.88> matha12
      }{}
\DeclareSymbolFont{matha}{U}{matha}{m}{n}
\DeclareMathSymbol{\odiv}         {2}{matha}{"63}


\newcommand\subsetsim{\mathrel{%
  \ooalign{\raise0.2ex\hbox{\scalebox{0.9}{$\subset$}}\cr\hidewidth\raise-0.85ex\hbox{\scalebox{0.9}{$\sim$}}\hidewidth\cr}}}
\newcommand\simsubset{\mathrel{%
  \ooalign{\raise-0.2ex\hbox{\scalebox{0.9}{$\subset$}}\cr\hidewidth\raise0.75ex\hbox{\scalebox{0.9}{$\sim$}}\hidewidth\cr}}}

\newcommand\simsubsetsim{\mathrel{%
  \ooalign{\raise0ex\hbox{\scalebox{0.8}{$\subset$}}\cr\hidewidth\raise1ex\hbox{\scalebox{0.75}{$\sim$}}\hidewidth\cr\raise-0.95ex\hbox{\scalebox{0.8}{$\sim$}}\cr\hidewidth}}}
\newcommand{\stcomp}[1]{{#1}^{\mathsf{c}}}

